\documentclass{article}
\usepackage[utf8]{inputenc}
\usepackage{amssymb,amsmath}
\usepackage[parfill]{parskip}
\DeclareMathOperator*{\argmin}{arg\,min}
\DeclareMathOperator*{\argmax}{arg\,max}
\usepackage{graphicx}
\usepackage{subcaption}

\usepackage{mathtools}
\DeclareMathOperator{\tr}{tr}

\title{Optimization 10/36-725\\
        Homework 2}
\author{Willie Neiswanger}
\date{}

\begin{document}

\maketitle


\section{Problem Two}

\textbf{(A)}  Let the SVD of $U V^\top + W = B S C^\top$. Then we have
\begin{flalign}
    &U V^\top + W = B S C^\top\\
    \iff &UV^\top = B S C^\top - W \\
    \iff &UV^\top V  = (B S C^\top - W)V\\
    \iff &U = B S C^\top V - WV &\rhd V \text{ orthogonal}\\
    \iff &U = B S C^\top V &\rhd WV=0\\
    \iff &B^\top U V^\top = B^\top B S C^\top V V^\top \\
    \iff &B^\top U V^\top = S C^\top  &\rhd B \text{ and } V \text{ orthogonal}\\
    \iff &B^\top U V^\top C = S C^\top C\\
    \iff &B^\top U V^\top C = S  &\rhd C \text{ orthogonal}\\
    \iff &\| B^\top U V^\top C \|_{op} = \|S\|_{op}
\end{flalign}
And $\| B^\top U V^\top C \|_{op}$ $\leq$ $\| B^\top\|_{op} \|U\|_{op}
\|V^\top\|_{op} \|C \|_{op}$ Each of $B$, $U$, $V$, and $C$ are rectangular
orthogonal matrices whose maximum singular values are bounded by 1. We can see
this with the following argument: Let $A$ be a rectangular orthogonal matrix
whose ``inner square" orthogonal matrix is written $\tilde{A}$ with
decomposition $\tilde{A} = \tilde{U} \tilde{\Sigma} \tilde{V}^\top$. We add
additional zeros to the diagonal of $\tilde{\Sigma}$ and orthogonal unit vector
rows to $\tilde{U}$ and $\tilde{V}^{\top}$, to get matrices $U$, $\Sigma$, and
$V^\top$, so that $U,\Sigma,V^\top$ $\in$ $\mathbb{R}^{n \times n}$ and $U
\Sigma V^\top = A$ (note that $U$, $\Sigma$, and $V^{\top}$ indeed exist and
their product equals $A$). The singular values of $\tilde{A}$ are all less then
1, so therefore the singular values of $A$ are also all less than 1.\\


\textbf{(B) } Using the properties of matrix products under the trace, we have
\begin{align}
        &\tr \left( (U V^\top + W)^\top A \right) \\
    =   &\tr \left( (U V^\top + W)^\top U \Sigma V^\top \right)\\
    =   &\tr \left( (U V^\top + W) V \Sigma U^\top \right)\\
    =   &\tr \left( (U + W V) \Sigma U^\top \right)\\
    =   &\tr \left( \Sigma + \Sigma U^\top W V \right)\\
    =   &\tr \left( \Sigma \right)
\end{align}


\textbf{(C) } Note that the dual of the trace norm is the operator norm, i.e.
$\|A\|_* = \max_{\|B\|_{op} \leq 1} \tr(B^\top A)$, and hence
\begin{align}
    \partial \|A\|_* &= \text{cl} \left[ \text{conv}\left(
        \bigcup_{\substack{\|B\|_{op} \leq 1 \\ \tr(B^\top A) = \|A\|_*}} \partial
        \tr(B^\top A)  \right) \right] \\
    &= \text{cl} \left[ \text{conv}\left( \bigcup_{\substack{\|B\|_{op} \leq 1
        \\ \tr(B^\top A) = \|A\|_*}} B \right) \right] \\
    &=  \text{cl} \left[ \text{conv} \{ B \hspace{1mm}:\hspace{1mm}
        \|B\|_{op} \leq 1, \tr(B^\top A) = \|A\|_* \} \right]\\
    &\supseteq \{ B \hspace{1mm}:\hspace{1mm} \|B\|_{op} \leq 1, \tr(B^\top A)
        = \|A\|_* \}
    %&= \{ U V^\top + W \hspace{1mm}:\hspace{1mm} \|W\|_{op} \leq 1, U^\top W =
        %0, W V = 0 \}
\end{align}
Let $B = U V^\top + W$ such that $\|W\|_{op} \leq 1$, $U^\top W = 0$, $WV = 0$.
We know then from parts (a) and (b) that this set of $B$ has the properties that
$\|B\|_{op}$ $=$ $\|U V^\top + W\|_{op}$ $\leq 1$ and $\tr(B^\top A)$ $=$
$\tr((U V^\top + W)^\top A)$ $=$ $\|A\|_*$, and therefore
\begin{align}
    \partial \|A\|_* &\supset \{ B \hspace{1mm}:\hspace{1mm} \|B\|_{op} \leq 1, \tr(B^\top A)
        = \|A\|_* \}\\
    &= \{ U V^\top + W \hspace{1mm}:\hspace{1mm} \|W\|_{op} \leq 1, U^\top W =
        0, W V = 0 \}
\end{align}


\textbf{(D) } 
We use a similar sequence of steps. Note that the dual of the operator norm is the trace norm, i.e.
$\|A\|_{op} = \max_{\|B\|_* \leq 1} \tr(B^\top A)$, and hence
\begin{align}
    \partial \|A\|_{op} &\supseteq \text{conv}\left( \bigcup_{\substack{\|B\|_* \leq 1
    \\ \tr(B^\top A) = \|A\|_{op}}} B \right)\\
    &\supseteq \text{conv}\{B : \|B\|_* \leq 1, \tr(B^\top A) = \|A\|_{op} \}
\end{align}
We can let $B = u_j v_j^\top$, where $j = \|A\|_{op}$, and show this is
equvalent to the above set. Note that $\|B\|_*$ $=$ $\|u_j v_j^\top\|_* \leq
1$, and also that $\tr(B^\top A) = \tr((u_j v_j^\top)^\top A) = \|A\|_{op}$,
because
\begin{align}
    \tr((u_j v_j^\top)^\top A) &= \tr(v_j u_j^\top A)\\
                               &= \tr(u_j^\top A v_j) \\
                               &= \tr\left(u_j^\top U \Sigma V^\top v_j \right)\\
                               &= \Sigma_{jj} = \|A\|_{op}
\end{align}
and hence
\begin{align}
    \partial \|A\|_{op} &\supseteq \text{conv}\{B : \|B\|_* \leq 1, \tr(B^\top A) = \|A\|_{op} \}\\
                        &= \text{conv}\{u_j v_j^\top : \Sigma_{jj} = \Sigma_{11} \}
\end{align}


\textbf{(E) }
Assuming this fact, the subdifferential $\partial \|A\|_{op}$ contains a single
element when there is only 1 maximum singular value for $A$ (since $A = U
\Sigma V^\top$, if $\Sigma$ had multiple maximum elements $\sigma_1, \sigma_2,
\ldots$ on its diagonal, then $\partial \|A\|_{op} = \{ u_{j_1} v_{j_1}^\top,
u_{j_2} v_{j_2}^\top, \ldots \}$). This implies that the operator norm of matrix A
if differentiable when $A$ has a unique maximum singular value.

\end{document}
