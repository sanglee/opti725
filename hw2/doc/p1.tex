\documentclass{article}
\usepackage[utf8]{inputenc}
\usepackage{amssymb,amsmath}
\usepackage[parfill]{parskip}
\DeclareMathOperator*{\argmin}{arg\,min}
\DeclareMathOperator*{\argmax}{arg\,max}
\usepackage{graphicx}
\usepackage[top=1in, bottom=1.25in, left=1.4in, right=1.4in]{geometry}


\title{Optimization 10/36-725, 
        Homework 2, Problem 1}
\author{}
\date{}

\begin{document}

\maketitle
\vspace{-6mm}

\textbf{(A)} First I will prove that $S$ is convex. Suppose not. Then $\exists$
$x,y \in S$, s.t. $z = \theta x + (1 - \theta) y$ $\not\in S$ (for $\theta \in
[0,1]$) $\implies$ $f(z) > f(x) = f(y)$ $\implies$ $f(\theta x + (1 -
\theta)y)$ $>$ $f(x) = f(y) = \theta f(x) + (1-\theta)f(y)$, which contradicts
the convexity of $f$, and thus $S$ is convex.

Next I'll prove, for strictly convex $f$, $S$ has one element. Suppose not.
Then $x \neq y \in S$ $\implies \theta x + (1 - \theta)y \in S$ (since $S$
convex by the above argument), $\implies f(\theta x + (1-\theta)y) = f(x) =
f(y) = \theta f(x) + (1-\theta)f(y)$, which contradicts the strict convexity of
$f$, and thus $S$ has a single element.\\


\textbf{(B)} They are all equal to 1, since the singular values of $A$ are the
eigenvalues of $A^{\top}A$, which equals (when $A$ orthogonal) $I_n$, whose
eigenvalues are all 1.

Consider $I_2$ and $-I_2$, where an example convex combination is $0.5 I_2 -
0.5I_2 = \mathbf{0}$, which is not orthogonal.\\


\textbf{(C)} For orthogonal $A_1,\ldots,A_n$, $\| \sum_{i=1}^n \theta_i A_i
\|_{op}$ $\leq$ $\sum_{i=1}^n \theta_i \| A_i \|_{op}$ $\leq \sum_{i=1}^n
\theta_i$ $= 1$ (where $\theta_i$s are coefficients of convex combination).

Let $S = \{A : \|A\|_{op} \leq 1 \}$. $\forall A_1, A_2 \in S$, $\|\theta A_1 +
(1-\theta)A_2\|_{op}$ $\leq$ $\theta \|A_1\|_{op} + (1-\theta)\|A_2\|_{op}$
$\leq \theta + (1-\theta)$ $=1$ $\implies$ $\theta A_1 + (1-\theta)A_2 \in S$
and S is convex.

By the reverse triangle inequality, $\|A-B\|_{op} \geq \left| \|A\|_{op} -
\|B\|_{op} \right|$ $\implies$ it is Lipschitz with $L=1$.\\


\textbf{(D)} Let $S$ be the unit spectral norm ball and $C$ the convex hull of
orthogonal matrices. Part (C) implies $C$ $\subset$ $S$. We will now show that
$S$ $\subset$ $C$: Let $A \in S$ and $A$ $=$ $U \Sigma V^{\top}$ (its SVD),
where we have added additional zeros to the diagonal of $\Sigma$ and orthogonal
unit vector rows to $U$ and $V^{\top}$, so that $U,\Sigma,V^{\top}$ $\in$
$\mathbb{R}^{n \times n}$ (note that $U$, $\Sigma$, and $V^{\top}$ indeed exist
and their product still equals $A$). Now, we can write $\Sigma =
\sum_{i=1}^{2^n} \theta_i (\tilde{I}_n)_i$, where $\sum_{i=1}^{2^n} \theta_i =
1$, and $\{\tilde{I}_n\}_{i=1}^{2^n}$ are all $n \times n$ diagonal matrices
containing only $1$s and $-1s$ ($\Sigma$ can be decomposed into this sum, since
we have a system of $n+1$ equations of $2^n$ variables, and thus there exists a
solution $\{\theta_i\}_{i=1}^{2^n}$). Hence, $A = U (\sum_{i=1}^{2^n} \theta_i
(\tilde{I}_n)_i)V^{\top}$ $=$ $\sum_{i=1}^{2^n} \theta_i U(\tilde{I}_n)_i
V^{\top}$. Since $U$, $V^{\top}$, and $(\tilde{I}_n)_i$ are all orthogonal,
their product is orthogonal, and thus $\|A\|_{op}$ $\leq$ $\sum_{i=1}^{2^n}
\theta_i \|U (\tilde{I}_n)_i V^{\top}\|_{op}$ $\leq$ $\sum_{i=1}^{2^n}
\theta_i$ $\leq$ $1$ $\implies$ $S \subset C$.\\


\textbf{(E)}
We know that $f(y) \geq f(x) + \nabla f(x)^\top(y-x) + \frac{1}{2}\lambda
\|y-x\|^2$ for all $x,y$, which implies that
\begin{align}
    &f(x) + f(y) \geq f(x) + f(y) + \nabla f(x)^\top(y-x) - \nabla f(y)^\top
        (y-x) + \lambda \|y-x\|^2\\
    &\iff 0 \geq \nabla f(x)^\top (y-x) - \nabla f(y)^\top(y-x) + \lambda
        \|y-x\|^2 \\
    &\iff (\nabla f(y) - \nabla f(x))^\top (y-x) \geq \nabla \|y-x\|^2
\end{align}
This implies that $\|\nabla f(y) - \nabla f(x)\| \|y-x\|$ $\geq$ $\lambda
\|y-x\|^2$ (by Cauchy-Schwarz). If we divide both sides by $\|y - x\|$, we get
the result. If, additionally, it's gradient is L-Lipschitz, we can say that
$\lambda \leq L$.

\end{document}
