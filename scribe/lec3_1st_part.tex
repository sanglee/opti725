\section{Pivot Transformation}
Consider the following LP problem

\begin{equation}
    \begin{split}
        \min z \hspace{4mm}\text{s.t.} \\
        2 x_1 + 2 x_2 + 2 x_3 + x_4 + 4 x_5 = z \\
        4 x_1 + 2 x_2 + 13 x_3 + 3 x_4 + x_5 = 17 \\
        x_1 + x_2 + 5 x_3 + x_4 + x_5 = 7 \\
        x_1, \ldots, x_5 \geq 0
    \end{split}
\end{equation}

We define a \textbf{pivot} to be a nonzero element in the above problem, for example
$3x_4$. We choose a pivot, then use it to eliminate the corresponding variable
(in this case, $x_4$) from the remaining equations. After we apply a pivot
transformation we arrive at an equivalent system of equations, where the
solution set is left the same. This process is equivalent to Gaussian
elimination. The pivot transformation procedes as follows.

We begin with an initial system of equations
\begin{eqnarray}
    \label{example01}
    \begin{aligned}
        && 2x_1 &&+&& 2x_2 &&+&& 2x_3 &&+&& x_4 &&+&& 4x_5 &&=&& z& \\
        && 4x_1 &&+&& 2x_2 &&+&& 13x_3 &&+&& 3x_4 &&+&& x_5 &&=&& 17& \\
        &&  x_1 &&+&& x_2 &&+&& 5x_3 &&+&& x_4 &&+&& x_5 &&=&& 7& \\
    \end{aligned}
\end{eqnarray}

We choose $3x_4$ as a pivot and aim to eliminate $x_4$ from the remaining equations. We begin by dividing the second equation by $3$ to get the system
\begin{eqnarray}
    \label{example01}
    \begin{aligned}
        && 2x_1 &&+&& 2x_2 &&+&& 2x_3 &&+&& x_4 &&+&& 4x_5 &&=&& z& \\
        && \frac{4x_1}{3} &&+&& \frac{2x_2}{3} &&+&& \frac{13x_3}{3} &&+&& x_4
            &&+&& \frac{x_5}{3} &&=&& \frac{17}{3}& \\
        &&  x_1 &&+&& x_2 &&+&& 5x_3 &&+&& x_4 &&+&& x_5 &&=&& 7& \\
    \end{aligned}
\end{eqnarray}

Note that the coefficients for $x_4$ in the first and third equations are now
the same as the coefficient for $x_4$ in the second equation. We then subtract
the the second equation from the first and third equations to get
\begin{eqnarray}
    \label{example01}
    \begin{aligned}
        && \frac{2x_1}{3} &&+&& \frac{4x_2}{3} &&-&& \frac{7x_3}{3} &&+&& 0x_4
            &&+&& \frac{11x_5}{3} &&=&& z - \frac{17}{3}& \\
        && \frac{4x_1}{3} &&+&& \frac{2x_2}{3} &&+&& \frac{13x_3}{3} &&+&& x_4
            &&+&& \frac{x_5}{3} &&=&& \frac{17}{3}& \\
        && -\frac{x_1}{3} &&+&& \frac{x_2}{3} &&+&& \frac{2x_3}{3} &&+&& 0x_4
            &&+&& \frac{2x_5}{3} &&=&& \frac{4}{3}& \\
    \end{aligned}
\end{eqnarray}

Note that now $x_4$ has been eliminated from equations one and three (i.e. the
coefficient for $x_4$ in these equations is now $0$). Next, we choose another
pivot, such as $\frac{x_2}{3}$. We use this pivot (in the third equation) to
eliminate $x_2$ from the first and second equations by subtracting $4$-times
the third equation from the first equation, to get
\begin{eqnarray}
    \label{example01}
    \begin{aligned}
        && 2x_1 &&+&& 0x_2 &&-&& 5x_3 &&+&& 0x_4 &&+&& x_5 &&=&& z -11& \\
        && \frac{4x_1}{3} &&+&& \frac{2x_2}{3} &&+&& \frac{13x_3}{3} &&+&& x_4
            &&+&& \frac{x_5}{3} &&=&& \frac{17}{3}& \\
        && -\frac{x_1}{3} &&+&& \frac{x_2}{3} &&+&& \frac{2x_3}{3} &&+&& 0x_4
            &&+&& \frac{2x_5}{3} &&=&& \frac{4}{3}& \\
    \end{aligned}
\end{eqnarray}
And then by subtracting $2$-times the third equation from the second equation,
to get 
\begin{eqnarray}
    \label{example01}
    \begin{aligned}
        && 2x_1 &&+&& 0x_2 &&-&& 5x_3 &&+&& 0x_4 &&+&& x_5 &&=&& z -11& \\
        && 2x_1 &&+&& 0x_2 &&+&& 3x_3 &&+&& x_4 &&-&& x_5 &&=&& 3& \\
        && -\frac{x_1}{3} &&+&& \frac{x_2}{3} &&+&& \frac{2x_3}{3} &&+&& 0x_4
            &&+&& \frac{2x_5}{3} &&=&& \frac{4}{3}& \\
    \end{aligned}
\end{eqnarray}

We also multiply the third equation by $3$ to get
\begin{eqnarray}
    \label{example01}
    \begin{aligned}
        && 2x_1 &&+&& 0x_2 &&-&& 5x_3 &&+&& 0x_4 &&+&& x_5 &&=&& z -11& \\
        && 2x_1 &&+&& 0x_2 &&+&& 3x_3 &&+&& x_4 &&-&& x_5 &&=&& 3& \\
        && -x_1 &&+&& x_2 &&+&& 2x_3 &&+&& 0x_4 &&+&& 2x_5 &&=&& 4& \\
    \end{aligned}
\end{eqnarray}

We can now rewrite this as
\begin{eqnarray}
    \label{example01}
    \begin{aligned}
        -z \hspace{4mm}+ && 2x_1 &&+&& 0x_2 &&-&& 5x_3 &&+&& 0x_4 &&+&& x_5 &&=&& -11& \\
        && 2x_1 &&+&& 0x_2 &&+&& 3x_3 &&+&& x_4 &&-&& x_5 &&=&& 3& \\
        && -x_1 &&+&& x_2 &&+&& 2x_3 &&+&& 0x_4 &&+&& 2x_5 &&=&& 4& \\
    \end{aligned}
\end{eqnarray}

The system written in this way is said to be in \textbf{canonical form}. In
particular, we say that this system of equations is in canonical form with
respect to $(-z)$, $x_4$, and $x_2$ variables, and that 
\begin{align*}
    x_1, x_3, x_5 &\text{ are independent (nonbasic) variables} \\
    -z, x_4, x_2 &\text{ are dependent (basic) variables, which are expressed
        in terms of other variables.}
\end{align*}

If we set the nonbasic variables to zero, then we can get values for the basic
variables. For example, setting $x_1=0$, $x_3=0$, and $x_5=0$, i.e. by setting
$X_N = (x_1,x_3,x_5) = (0,0,0)$ we get that $z=11$, $X_B = (x_4,x_2) = (3,4)$.

Note that if we had initially started with a different pivot, say $x_1$ and
then $x_4$ (instead of $x_4$ and then $x_2$), then we would've arrived at
$z=3$, $X_N = (x_2,x_3,x_5) = (0,0,0)$, and $X_B = (x_1,x_4) = (-4,11)$.


\subsection{The Goals of Pivots}
The primary goal of the pivoting procedure is to reduce the original LP problem
to canonical form. After a reduction to canonical form, it is easy to find a
(basic) solution to the system of equations. To do so, we simply set the
nonbasic variables to zero. Note that this procedure returns a valid solution,
because the pivoting process does not alter the solution set of the system of
equations (i.e. after pivots, the systems are equivalent).

However, the basic solution returned by the pivoting process might be:
\begin{enumerate}
    \item Not a feasible solution. This could occur because of the boundary
        constraint (e.g. we required that $x_1,\ldots,x_5>0$, in the case of the
        previous example).
    \item Not an optimal solution. For example, $z$ might not be minimal.
\end{enumerate}


\subsection{Formal Definition of Canonical Form}
A system of $m$ equations and $n$ nonbasic variables $(x_j)_{j=1}^n$ is in
\textbf{canonical form} with respect to $m$ basic variables $(x_{j_i})_{i=1}^m$
if and only if either of the two equivalent conditions hold:
\begin{enumerate}
    \item $x_{j_i}$ has one coefficient in equation $i$, for all $j \neq i$.
    \item The system $I_m X_B + A X_N = b$
\end{enumerate}

Using the previous example, the second condition would be equivalent to
\begin{align*}
    M = 
    \begin{bmatrix}
        1 & 0 & 0\\
        0 & 1 & 0\\
        0 & 0 & 1
    \end{bmatrix}
    \begin{bmatrix}
        -z\\
        x_4\\
        x_2
    \end{bmatrix}
    +
    \begin{bmatrix}
        2   &   -5  &   1   \\
        2   &   3   &   -1  \\
        -1  &   2   &   2 
    \end{bmatrix}
    \begin{bmatrix}
        x_1\\
        x_3\\
        x_5
    \end{bmatrix}
    =
    \begin{bmatrix}
        -11 \\
        3   \\
        4 
    \end{bmatrix}
\end{align*}


We more-formally define a \textbf{basic solution} to be the solution gained by
setting $X_N=0$ in the canonical form.

For example, if we let
\begin{align*}
    X_B = (x_1,\ldots,x_m)\\
    X_N = (x_{m+1},\ldots,x_n)
\end{align*}
then we'd set $x_{m+1},\ldots,x_n = 0$, and would arrive at the solutions
$x_1=b_1,\ldots,x_m=b_m$. This basic solution is feasible if and only if $b_1
\geq 0,\ldots,b_m \geq 0$.
